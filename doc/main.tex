\documentclass[a4paper,12pt,notitlepage]{mwrep}
%%\documentclass[polish,11pt,a4paper,twosides]{article}
%%%\usepackage{fullpage}

\usepackage{./mystyle}

\newcommand\bigforall{\mbox{\Large $\mathsurround0pt\forall$}} 
\begin{document}

% na potrzeby strony tytułowej robimy taki nagłówek : 
\newcommand{\cfoottext}{Kraków, październik 2012}
\renewcommand{\footrulewidth}{0.4pt}
\cfoot{\cfoottext}

\begin{titlepage}
\begin{center}
  \resizebox{\textwidth}{!}{\mbox{AKADEMIA GÓRNICZO-HUTNICZA}}\\
  \vspace{2ex}
  \resizebox{\textwidth}{!}{\mbox{Wydział Informatyki, Elektroniki i Telekomunikacji}}\\
  \vspace{4ex}
  \includegraphics[scale=0.15]{images/agh_crop.pdf} \\
  \vspace{4ex}
  \begin{large}KATEDRA INFORMATYKI\end{large} \\
  \vspace{8ex}
  \textbf{\begin{Huge}Problem plecakowy\end{Huge}} \\
  \vspace{3ex}
  \textit{\begin{Large}Algorytm pszczeli\end{Large}} \\

  \vfill

  \begin{Large}\wersja\end{Large}\\
\end{center}
\begin{large}
  \begin{tabularx}{\textwidth}{lXc}
    \textit{Kierunek, rok studiów} &  & \textit{ } \\
    \hspace{3em}Informatyka, rok III &  &  \\
    \textit{Przedmiot} &  & \\
	\multicolumn{3}{l}{\hspace{3em}Badania Operacyjne} \\
    \textit{Prowadzący przedmiot} &  & \textit{rok akademicki:} \hfill 2012/2013\\
    \hspace{3em}dr inż. Joanna Kwiecień &  & \textit{semestr:} \hfill zimowy\\
  \end{tabularx}
\end{large}

\vspace{2ex}

\noindent
\begin{Large}
Skład zespołu:\end{Large}
\begin{large}
  \newlength{\tblwidth}
  \setlength{\tblwidth}{\textwidth}
  \addtolength{\tblwidth}{-3em}
  \begin{flushright}
    \begin{tabularx}{\tblwidth}{lXr}
      Martyna Wałaszewska &  &  \\
      Rafał Szalecki &  &  \\
      Wojciech Krzystek &  &  \\
      Grzegorz Wilaszek &  &  \\
    \end{tabularx}\end{flushright}
  \end{large}
  \thispagestyle{fancy}
\end{titlepage}



\onehalfspacing

%\rhead{\scriptsize{Maria}}
\lhead{\scriptsize{Wałaszewska, Szalecki, Krzystek, Wilaszek}}

\lfoot{\scriptsize{Copyright \textcopyright 2012 AGH}}
\cfoot{\normalsize Strona \thepage \ z \pageref{LastPage}}
\rfoot{\scriptsize{\wersja}}
\setcounter{secnumdepth}{2}

\setcounter{tocdepth}{2}


\vfill
\begin{center}
\singlespacing
\fbox{\begin{minipage}{0.8\textwidth}
\footnotesize Niniejsze opracowanie powstało w trakcie i jako rezultat zajęć dydaktycznych z przedmiotu 
wymienionego na stronie tytułowej, prowadzonych w Akademii Górniczo-Hutniczej w Krakowie
(AGH) przez osobę (osoby) wymienioną (wymienione) po słowach "Prowadzący zajęcia"
i nie może być wykorzystywane w jakikolwiek sposób i do jakichkolwiek celów, 
w całości lub części, w szczególności publikowane w jakikolwiek sposób
i w jakiejkolwiek formie, bez uzyskania uprzedniej, pisemnej zgody tej
osoby (tych osób) lub odpowiednich władz AGH.
\vspace{2ex} \\
\textbf{Copyright \textcopyright 2012 Akademia Górniczo-Hutnicza (AGH) w Krakowie}
      \end{minipage}
}
\onehalfspacing
\end{center}

\tableofcontents

\chapter{Tematyka projektu}

\chapter{Opis problemu}
\section{Definicja}
W ramach projektu zajmiemy się rozwiązywaniem wielowymiarowego problemu plecakowego.
Problem ten zdefiniowany jest następująco.

Dany jest zbiór $n$ przedmiotów. $j$-ty przedmiot $1 \le j \le n$
opisany jest przez swoją wartość~$p_j$ oraz ciąg wag
$(w_{1j}, w_{2j}, \dots, w_{mj})$, gdzie $m$ to liczba wymiarów.\\
Ponadto dany jest ciąg ogarniczeń plecaka $(W_1, W_2, \dots, W_m)$.\\
Zadanie polega na zmaksymalizowaniu sumy
$$ \sum_{j=1}^n p_j\chi_j$$
przy jednoczesnym zachowaniu nierówności
$$\bigforall_{1 \le i \le m}\quad\sum_{j=1}^n w_{ij}\chi_j \le W_i$$
gdzie $\chi_j \in \{0, 1\}$, wartość $\chi_j=1$
oznacza użycie $j$-tego przedmiotu.

\section{Wyjaśnienie}
Nie należy mylnie utożsamiać wymiarów w problemie plecakowym z 
wymiarami przestrzeni. Intuicyjnie problem można rozumieć jako zadanie
zapakowania plecaka tak, aby sumaryczna wartość (cena)
przedmiotów była jak największa.
Przy pakowaniu należy uwzględnić $m$ \textbf{niezależnych od siebie}
ograniczeń. W rzeczywistości takimi ogarniczeniami mogą być np. waga,
objętość, zawartość nadsiarczanu potasu.

Przedmiot jest opisany przez
rozmiar w każdym z wymiarów oraz wartość (cenę). Ograniczenie plecaka
$W_i$ oznacza, że suma rozmiarów zapakowanych przedmiotów w $i$-tym
wymiarze nie może przekroczyć $W_i$.

Analizowany problem nie uwzględnia położenia przedmiotów w plecaku
ani ich geometrycznych własności. Jest to konsekwencja tego, że każdy
wymiar rozpatrujemy osobno. Dlatego nie można myśleć o rozmiarach
przedmiotu jak o rozmiarach geometrycznych.

\chapter{Algorytm pszczeli}
\section{Opis idei algorytmu}
\section{Implementacja rozwiązująca problem plecakowy}

\chapter{Opis aplikacji}
\section{Podręcznik użytkownika}
\subsection{Instalacja}

\chapter{Podsumowanie}

\chapter{Podział pracy}

\addtocounter{page}{-1}

\appendix
\chapter*{Załączniki}
\begin{description}
	\item[plik blah.pdf]	 --- blahblah - Pierwszy załącznik
\end{description}

\begin{thebibliography}{9}

\bibitem{enwiki}
	\href{http://en.wikipedia.org/wiki/List_of_knapsack_problems}{http://en.wikipedia.org/wiki/List\_of\_knapsack\_problems}

\end{thebibliography}


\listoffigures

\listoftables

\label{LastPage}\phantom{\phantomsection{LastPage}}
\end{document}
